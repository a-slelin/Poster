\documentclass{a0poster}
\usepackage{fancytikzposter} 
%%Сеты
\setmargin{1}
\setblockspacing{1}
\setcolumnnumber{2}
\usetemplate{1}

\definecolor{myblue}{HTML}{008888} 
\setthirdcolor{white!80!black}
\usepackage[margin=\margin cm, paperwidth=84.1cm, paperheight=118.9cm]{geometry}

\usepackage{cmbright}
\usepackage[math]{kurier}
\usepackage[utf8]{inputenc}
\usepackage[english,russian]{babel}
\usepackage{graphicx}
\usepackage{hyperref}
\usepackage{setspace}
\newcommand{\eps}{\varepsilon}

\title{Подпространства и ранг}
\author
{
Слелин А. В. \\
a.slelin.work@mail.ru \\
ЯрГУ им. П. Г. Демидова\\
}

\begin{document}

\ClearShipoutPicture
\AddToShipoutPicture{\BackgroundPicture}
\noindent 
\begin{tikzpicture}
  \initializesizeandshifts
  \ifthenelse{\equal{\template}{1}}
  { 
    \titleblock{51}{1}
  }
  {
    \titleblock{51}{1.5}
  }
  \addlogo[south west]{(2,2)}{5cm}{yarsu_logor.png}



\calloutblock{(0,43)}{(0,43)}{78cm}
{
  \begin{flushleft}
      \textbf{\Large Подпространства линейного пространства. Примеры. Ранг и база системы векторов.}
  \end{flushleft}

  \vspace{0.6cm}
  Пусть $L$ -- произвольное линейное пространство.\newline
  {\bf Опpеделение 1.} Непустое подмножество $L_{1} \subset L$ называется \emph{линейным подпространством}, если $L_{1}$ замкнуто относительно операций сложения и умножения на число:\newline
  \textbf{1)} $\forall x, y \in L_{1} \quad x + y \in L_{1};$ \newline
  \textbf{2)} $\forall x \in L_{1} $ и $\forall \alpha \in \textnormal{R} \quad \alpha x \in L_{1}.$ 

  \vspace{0.3cm}
  \textbf{Замечание:} из второго условия при $\alpha = 0$ сразу следует то, что $0 \in L_{1}$. Таким образом, все линейные подпространства одного пространства $L$ обязательно содержат общий элемент, а именно нулевой вектор $L$. Подмножество не содержащее нуля, линейным подпространством не является.

  \vspace{0.3cm}
  \textbf{Примеры:}
  
  \vspace{0.3cm}
  \textbf{1)} Линейным пространством являются $\{0\}$ и всё $L$. Они называются \emph{несобственными подпространствами} $L$.
  
  \vspace{0.3cm}
  \textbf{2)} Собственные подпространства $V_{3}$ -- прямые и плоскости, проходящие через т.O.
  
  \vspace{0.3cm}
  \textbf{3)} Совокупность $L_{1}$ $n$-мерных векторов $x = (x_{1}, \dots, x_{n})$ таких, что $\sum\limits_{i=1}^{n}x_{i} = 0$, образуют линейное подпространство в $L = \textnormal{R}^{n}$.
  
  \vspace{0.3cm}
  \textbf{4)} Пусть $\textbf{A} \in M_{m, n}$ -- фиксированная матрица. Обозначим через $L_{1}$ подмножество пространства $L = \textnormal{R}^{n}$, состоящее из всех $x = (x_{1}, \dots, x_{n})$ таких, что 
  \[\tag{1}
    \textbf{A}
    \begin{pmatrix}
     x_{1}\\
     \vdots \\
     x_{n}
    \end{pmatrix} = 
    \begin{pmatrix}
        0\\
        \vdots\\
        0\\
    \end{pmatrix}.
\]
   $L_{1}$ является линейным подпространством.
   
   \vspace{0.3cm}
  \textbf{5)} Совокупности многочленов $\textnormal{R}_{j}[t],\; j = 0, 1, \dots, n - 1,$ степени $\leq j$ составляют цепочку собственных линейных подпространств одного пространства $\textnormal{R}_{n}[t]$.
  
  \vspace{0.3cm}
  \textbf{6)} Каждое из функциональных пространств $C^{\infty}[a, b],\; C^{k}[a, b],\; C[a, b]$ является бесконечномерным подпространством пространства $B[a, b]$ ограниченных на $[a, b]$ функций.
  
  \vspace{0.3cm}
  \textbf{7)} Каждое из пространств последовательностей $l_{1},\; c_{0},\; c$ является бесконечномерным линейным подпространством пространства $b = l_{\infty}$ ограниченных последовательностей.
  
  \vspace{0.3cm}
  \textbf{8)} Линейная оболочка $L_{1} = \textnormal{lin}(x_{1},\dots, x_{k})$ является конечномерным линейным подпространством $L$.
  
  \vspace{0.3cm}
  {\bf Опpеделение 2.} Размерность подпространства $L_{1} = \textnormal{lin}(x_{1},\dots, x_{k})$ называется \emph{рангом системы векторов} $x_{1},\dots, x_{n}$, а базис $L_{1}$, состоящий из каких-то векторов $x_{i}$, называется \emph{базой системы} $x_{1},\dots,x_{k}$.
} 



\plainblock[0]{($(20, 9.5)$)}{38}{Ранг матрицы. Теорема о ранге. Свойства ранга \\ матрицы.} 
{

\vspace{0.6cm}
\textbf{Опpеделение 5.} \emph{Рангом матрицы} $\textbf{A}$ называется ранг системы её столбцов как элементов $\textnormal{R}^{m}$, то есть размерность линейной оболочки системы столбцов $X_{1}, \dots, X_{n}$: $$\textnormal{rg}(\textbf{A})\; := \; \textnormal{rg} (X_{1}, \dots, X_{n})\; =\; \dim \; \textnormal{lim}(X_{1}, \dots, X_{n}).\eqno{(5)}$$

\vspace{0.3cm}
\textbf{Теорема 3.} Ранг матрицы равен максимальному порядку \emph{r} отличного от нуля минора этой матрицы.

\vspace{0.3cm}  
\textbf{Следствие 1.} Для каждой $\textbf{A} \in M_{m, n} \quad \textnormal{rg}(\textbf{A}^{\textbf{T}}) = \textnormal{rg}(\textbf{A}).$

  
\vspace{0.3cm}
\textbf{Свойства:} 

\vspace{0.3cm} 
\textbf{1)} Для $\textbf{A} \in M_{m, n} \quad \textnormal{rg}(\textbf{A}) \leq \min (m, n)$.

\vspace{0.3cm} 
\textbf{2)} Пусть $\textbf{A} \in M_{m, n}.$ $\textnormal{rg}(\textbf{A}) = n \Longleftrightarrow |\textbf{A}| \neq 0.$

\vspace{0.3cm} 
\textbf{3)} Пусть $\textbf{A} \in M_{m, n}.$ Матрица $\textbf{A}$ обратима $\Longleftrightarrow \textnormal{rg}(\textbf{A}) = n $.

\vspace{0.3cm} 
\textbf{4)} Если $\textbf{AB}$ существует, то $\textnormal{rg}(\textbf{AB}) \leq \min \big(\textnormal{rg}(\textbf{A}), \textnormal{rg}(\textbf{B})\big)$.

\vspace{0.3cm} 
\textbf{5)} Пусть $\textbf{B} \in M_{m, n}$ и $\textnormal{rg}(\textbf{B}) = n $. Если $\textbf{AB}$ существует, то $\textnormal{rg}(\textbf{AB}) = \textnormal{rg}(\textbf{A})$. То же -- для произведения $\textbf{BA}$.

\vspace{0.3cm} 
\textbf{6)} Для $\textbf{A} \in M_{m, k},\; \textbf{B} \in M_{k, n} \quad \textnormal{rg}(\textbf{A}) + \textnormal{rg}(\textbf{B}) \leq \textnormal{rg}(\textbf{AB}) + k$.

\vspace{0.3cm} 
\textbf{7)} Для $\textbf{A},\textbf{B} \in M_{m, n} \quad \textnormal{rg}(\textbf{A + B}) \leq  \textnormal{rg}(\textbf{A}) +  \textnormal{rg}(\textbf{B})$.

\vspace{0.3cm} 
\textbf{8)} Если все произведения существуют, то $$ \textnormal{rg}(\textbf{AB}) +  \textnormal{rg}(\textbf{BC}) \leq  \textnormal{rg}(\textbf{B}) + \textnormal{rg}(\textbf{ABC})\eqno{(6)}$$.
  \vspace{-0.41cm}
}








\plainblock[0]{($(20, -21.5)$)}{38}{Применение понятия ранга к анализу систем \\ линейных уравнений. Теорема Кронекера -- Капелли. \\ Критерий определённости.}
{ 

 \vspace{0.6cm}
 Пусть дана система $m$ уравнений с $n$ неизвестными $x_{1}, \dots, x_{n}$ и матрицей коэффициентов $\textbf{A} = (a_{i, j}) \in M_{m, n}$: \begin{center}
 \begin{equation} \tag{7}
\begin{aligned} 
     &a_{11}x_{1} \; + \; \dots \; + a_{1n}x_{n} \; = \; b_{1} \\
     &a_{21}x_{1} \; + \; \dots \; + a_{2n}x_{n} \; = \; b_{2} \\
     &\;\; \dots \quad \quad \;  \dots \quad \quad \; \dots \quad \quad \;\dots\\
     &a_{m1}x_{1} \; + \; \dots \; + a_{mn}x_{n} \; = \; b_{m} \\
 \end{aligned} \end{equation}\end{center}
 Пусть $X_{1},\dots,X_{n}$ -- столбцы матрицы $\textbf{A}$, $\textbf{b}$ -- столбец свободных членов. Обозначим через $\textbf{A}|\textbf{b}$ расширенную матрицу системы (7).
 
 \vspace{0.3cm}
 \textbf{Теорема 4.} Система (7) является совместной тогда и только тогда, когда $$\textnormal{rg}(\textbf{A}|\textbf{b}) = \textnormal{rg}(\textbf{A}).\eqno{(8)}$$
 
 \vspace{0.3cm}
 \textbf{Теорема 5.} Система линейных уравнений (7) является определённой тогда и только тогда, когда выполняются одновременно два равенства $$\textnormal{rg}(\textbf{A}|\textbf{b}) = \textnormal{rg}(\textbf{A}) = n.\eqno{(9)}$$ 

 \vspace{0.1cm}
}




\plainblock[0]{($(-20,-13)$)}{38}{Сумма и пересечение подпространств. Теорема о \\размерностях суммы и пересечения.}
{  

 \vspace{0.6cm}
 {\bf Опpеделение 4.} \emph{Сумма} и \emph{пересечение} подпространств $L_{1}$, $L_{2}$ линейного пространства $L$ определяется следующим образом: $$L_{1} + L_{2} := \{x \in L \;:\; x = x_{1} + x_{2}, \;x_{i} \in 
 L_{i},\; i = 1, 2 \},\eqno{(2)}$$ $$L_{1} \cap L_{2} := \{x \in L\; :\; x \in L_{i},\; i = 1, 2 \}.\eqno{(3)}$$

 \vspace{0.3cm}
 \textbf{Теорема 2.} $L_{1} + L_{2}$, $L_{1} \cap L_{2}$ -- линейный подпространства $L$. Если основное пространство $L$ конечномерно, то имеет место равенство $$\dim (L_{1} + L_{2}) + \dim L_{1} \cap L_{2} = \dim L_{1} + \dim L_{2}.\eqno{(4)}$$

 \vspace{0.3cm}
}









\plainblock[0]{($(-20,9.5)$)}{38}{Прямая сумма подпространства. Теорема о прямой \\ сумме.}
{  

   \vspace{0.4cm}
   Пусть $L$ -- произвольное линейное пространство, $L_{1}$, $L_{2}$ -- два его подпространства. Если их сумма $S := L_{1} + L_{2}$ обладает одним из нескольких эквивалентных свойств, то эта сумма является прямой.

   \vspace{0.2cm}
   {\bf Опpеделение 3.} Сумма $S = L_{1} + L_{2}$ называется \emph{прямой}, если для любого $x \in S$ представление $x = x_{1} + x_{2}$, $\;x_{1} \in L_{1}$, $\;x_{2} \in L_{2}$, является единственным.

   \vspace{0.2cm}
   \textbf{Теорема 1.} Сумма $S = L_{1} + L_{2}$ является прямой тогда и только тогда, когда выполнено любое из следующих эквивалентных условий.
   
   \vspace{0.2cm}
   \textbf{1)} $\; L_{1} \cap L_{2} = \{0\}.$

   \vspace{0.3cm}
   \textbf{2)} $\; \dim(L_{1} + L_{2}) = \dim L_{1} + \dim L_{2}.$

   \vspace{0.3cm}
   \textbf{3)} $\; \textnormal{Если } f_{1}, \dots, f_{l} \textnormal{ -- базиc } L_{1},\; g_{1}, \dots, g_{m} \textnormal{ -- базис } L_{2},  \textnormal{то } f_{1}, \dots, f_{l}, \;g_{1}, \dots, g_{m} \textnormal{ --} \textnormal{ базис } L_{1} + L_{2}.$

   \vspace{0.3cm}
   \textbf{4)} Единственность разложения по $ L_1 \textnormal{ и } L_{2} \textnormal{ имеет место для нулевого вектора: если } x_{1} + x_{2} = 0,\; x_{1} \in L_{1},\; x_{2} \in L_{2}, \textnormal{то обязательно } x_{1} = x_{2} = 0.$

   \vspace{0.3cm}
   \textbf{Утверждение 1.} Каждая матрица $\textbf{A} \in M_{n}$ единственным образом представляется в виде суммы симметричной $\textbf{B}\; (b_{ji} = b_{ij})$ и кососимметричной $\textbf{C}\;(c_{ji} = -c_{ij})$ матриц.
 }



\plainblock[0]{($(-20, -31)$)}{38}{Размерность и базис подпространства $\textnormal{R}^{n}$, задаваемого \\ системой линейных однородных уравнений.}
{ 

\vspace{0.3cm}
Рассмотрим систему линейных однородных уравнений (1) c данной матрицей коэффициентов $\textbf{A} \in M_{m, n}$. Пусть $L \in \textnormal{R}^{n}$ определяется равенством $$L := \{x = (x_{1}, \dots, x_{n})\; : \;x \textnormal{ удовлетворяет } (1)\}.\eqno{(10)}$$


\vspace{0.3cm}
\textbf{Теорема 6.} $\dim L = n - \textnormal{rg}(\textbf{A}).$

\vspace{0.4cm}
С каждой матрицей $\textbf{A} \in M_{m, n}$ можно связать два линейных подпространства $\textnormal{R}^{n}$:

\vspace{0.4cm}
\textbf{1)} $L_{1}$ -- линейная оболочка строк матрицы $\textbf{A}$. Размерность $\dim L_{1} = \textnormal{rg}(\textbf{A})$. Базис $L_{1}$ образует любая система из $r = \textnormal{rg}(\textbf{A})$ линейно независимых строк.

\vspace{0.4cm}
\textbf{2)} $L_{2}$ -- подпространство решений системы однородных уравнений. Размерность $\dim L_{2} = n - \textnormal{rg}(\textbf{A})$. Базис $L_{2}$ образует фундаментальную систему решений данной системы уравнений.

\vspace{0.3cm}
}



 

\plainblock[0]{($(0,-50)$)}{78}{Литература}
{   
\begin{enumerate}
\bibitem{S1}
{\it Невский М. В.} Подпространства и ранг. // Лекции по алгебре: Учебное пособие // Яpославль: ЯрГУ, 2002. с. 72 - 85
\end{enumerate}
}
  
\end{tikzpicture}

\end{document}
